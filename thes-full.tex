%!TEX root=/home/ska124/Dropbox/Thesis/thes-full.tex
%% Copyright 1998 Pepe Kubon
%%
%% `thes-full.tex' --- the example thesis, FULL version, used
%%                     with  the `csthesis' package 
%% Use: latex thes-full to generate the DVI output, then 
%%      bibtex thes-full to generate the bibliography
%%      makeindex thes-full to get the index, and
%%      latex thes-full (2x) 
%%
%% You are allowed to distribute this file together with all files
%% mentioned in READ.ME.
%%
%% You are not allowed to modify its contents.
%%

\documentclass[11pt]{report}
%\documentclass[11pt,twoside]{report}%% for two-sided printing
\usepackage{pdfpages} 
\usepackage{anysize,fancyhdr,graphics}
\usepackage{csthesis}
\usepackage{makeidx}  %%% standard INDEX

\usepackage[titletoc]{appendix}%%Ensure word appendix appears in toc

%% Custom Packages 

\usepackage{subfig,graphicx}
\usepackage{array}
\usepackage{enumitem}
\usepackage{listings}


\makeindex  

%%% The following code demonstrates the ``other list'' facility. A new
%%% command \otherlist is defined for the List of Programs. Programs
%%% are defined as floating environments of type 3 (1 is used for figures,
%%% 2 for tables) and the information about them is stored in an
%%% auxiliary file with .lop extension. You can use this method to
%%% define several types of ``other lists,'' but in that case you'll
%%% need to add appropriate code to \lists in the csthesis.sty
%%% package.
%%% Note: It's better to move this code into your own mythesis.sty
%%% package. If you do that, you should get rid of the \makeatletter,
%%% \makeatother commands.
\makeatletter
\newcommand\otherlist{%
    \addcontentsline{toc}{chapter}{\otherlistname}
    \if@twocolumn
      \@restonecoltrue\onecolumn
    \else
      \@restonecolfalse
    \fi
    \chapter*{\otherlistname
      \@mkboth{\MakeUppercase\otherlistname}%
              {\MakeUppercase\otherlistname}}%
    \@starttoc{lop}%
    \if@restonecol\twocolumn\fi
    }
\newcommand*\l@program{\@dottedtocline{1}{1.5em}{2.3em}}
\newcommand\otherlistname{List of Programs}
\newcommand\programname{Program}
\newcounter{program}[chapter]
\renewcommand\theprogram{\thechapter.\@arabic\c@program}
\def\fps@program{tbp}
\def\ftype@program{3}
\def\ext@program{lop}
\def\fnum@program{\programname~\theprogram}

% Amoeba Cache
\def\AC{\textit{Amoeba-Cache}}
\def\AB{\textit{Amoeba-Block}}

%%%%%%%%%%%%%%%%%%%%%%%%%%%%%%%%%%%%%%%%%%%%%%%%%%%%%%%%%%%%%%%%%%%%%%
% mls: \code{...} command.
% Like \verb|...| but with braces, and not so fragile.
%
% Give me back the <, >, and _ characters in chosen modes
% unfortunately, this hack does not work when the <, >, or _ is embedded
% in another command.  In such circumstances use $<$, $>$, and \_
%
\makeatletter
\def\real@ltgtus{%
    \catcode`<=\active
    \catcode`>=\active
    \catcode`\_=\active
}
{\real@ltgtus
    \gdef<{\futurelet\@let@token\less@than}%
    \gdef>{\futurelet\@let@token\greater@than}%
    \gdef_{\underscore}%
}
% modify \textunderscore (standard LaTeX macro) to print as the _
% character in \tt font; as an appropriate rule in other fonts.
\renewcommand{\textunderscore}{\ifdim\fontdimen4\font=0pt\string_\else
    \leavevmode\kern.06em\vbox{\hrule width0.3em}\fi}
% \underscore is subscript in math mode, textunderscore otherwise
\DeclareRobustCommand{\underscore}{\ifmmode\sb\else\textunderscore\fi}
% similarly, create backslash, lessthan, and greaterthan macros that use
% the proper font:
\def\bs{\ifdim\fontdimen4\font=0pt\char92\relax\else
    \leavevmode$\backslash$\fi}
\newcommand{\less@than}{\ifdim\fontdimen4\font=0pt\string<\else
    \leavevmode\mathhexbox13C\fi}
\newcommand{\greater@than}{\ifdim\fontdimen4\font=0pt\string>\else
    \leavevmode\mathhexbox13E\fi}

% \code prints its argument in fixed-width font.
% There are no special characters in a code command, other than braces
% and backslash.
% It's similar to \verb, except that it's delimited normally (with
% braces).
\def\verythinspace{\kern .05em }
\DeclareRobustCommand{\code}{\begingroup
    \frenchspacing
    \real@ltgtus
    \@makeother\$\@makeother\&\@makeother\#%
    \@makeother\^\@makeother\%\@makeother\~%
    \@code}
\let\codefont\tt
\def\@code#1{\strut\verythinspace{\codefont
    #1}\verythinspace\strut\endgroup}
% For some reason I don't understand, escaped curly braces don't work
% right in \code commands.  Use the following instead:
\def\ttlb{{\tt\char123}}
\def\ttrb{{\tt\char125}}
\def\ttcaret{{\tt\char94}}
\def\tttilde{{\tt\char126}}
\makeatother
%%%%%%%%%%%%%%%%%%%%%%%%%%%%%%%%%%%%%%%%%%%%%%%%%%%%%%%%%%%%%%%%%%%%%%

%% Spots --- The round black magic blob

\def\spot#1{\begin{picture}(10,10)(-5,-4)%                                      
    \put(1,0){\circle*{10}}\put(-1,-3){\textcolor{white}{#1}}\end{picture}}

\def\bigspot#1{\begin{picture}(15,15)(-5,-5)%                                   
    \put(2,0){\circle*{15}}\put(-1,-3){\textbf{\textcolor{white}{#1}}}\end{picture}}



\newenvironment{program}
               {\@float{program}}
               {\end@float}
\newenvironment{program*}
               {\@dblfloat{program}}
               {\end@dblfloat}
\makeatother
%%% end of ``other list'' code

\begin{document}
\setlength{\pdfpagewidth}{8.5in}
\setlength{\pdfpageheight}{11in}
%%% set switches
%\contentspagefalse  
\figurespagetrue
\tablespagetrue
\dedicationpagetrue
\quotationpagetrue
% \otherlistpagetrue

%%% front matter 
\input{files/00-titapp} %% title, approval

%% Partial Copyright License (PCL)
\newpage 			
\addcontentsline{toc}{chapter}{Partial Copyright License}
\mbox{}
\makeatletter
\AddToShipoutPictureBG*{
            \setlength{\@tempdimc}{.06\paperheight}
            \setlength{\unitlength}{1pt}
           \put(\strip@pt\@tempdimb,\strip@pt\@tempdimc){
	\includegraphics{PCL_Declaration_2011.pdf}
	} 
} 
\makeatother
\newpage

%!TEX root=/home/ska124/Dropbox/Thesis/thes-full.tex
%% Copyright 1998 Pepe Kubon
%%
%% `abstract.tex' --- abstract for thes-full.tex, thes-short-tex from
%%                    the `csthesis' bundle
%%
%% You are allowed to distribute this file together with all files
%% mentioned in READ.ME.
%%
%% You are not allowed to modify its contents.
%%

%%%%%%%%%%%%%%%%%%%%%%%%%%%%%%%%%%%%%%%%%%%%%%%%%
%
%       Abstract 
%
%%%%%%%%%%%%%%%%%%%%%%%%%%%%%%%%%%%%%%%%%%%%%%%%

\prefacesection{Abstract}

Memory in modern computing systems are hierarchical in
nature. Maintaining a memory hierarchy enables the system to service
frequently requested data from a small low latency store located close
to the processor. The design paradigms of the memory hierarchy have
been mostly unchanged since their inception in the late
1960's. However in the meantime there have been significant changes in
the tasks computers perform and the way they are programmed. Modern
computing systems perform more data centric tasks and are programmed
in higher level languages which introduce many layers of abstraction
between the programmer and the system. \\ 

Waste in the memory hierarchy refers to the under utilized space in
the memory system and consequently wasted energy and time. The data
access patterns of modern workloads are increasingly less uniform
which makes it hard to design a memory hierarchy with rigid design
principles that performs optimally for a wide range of workloads.  The
problem is exacerbated by the implications of the growing fraction of
dark silicon on a processor chip. \\

This dissertation proposes and evaluates the benefits of a novel
architecture for the on chip memory hierarchy which would allow it to
dynamically adapt to the requirements of the application. We propose a
design that can support a variable number of cache blocks, each of a
different granularity. It employs a novel organization that completely
eliminates the tag array, treating the storage array as uniform and
morph-able between tags and data.  This enables the cache to harvest
space from unused words in blocks for additional tag storage, thereby
supporting a variable number of tags (and correspondingly, blocks).
The design adjusts individual cache line granularity according to
the spatial locality in the application.  It adapts to the appropriate
granularity both for different data objects in an application as well
as for different phases of access to the same data. \\

Compared to a fixed granularity cache, improves cache utilization to
90\% - 99\% for most applications, saves miss rate by up to 73\% at
the L1 level and up to 88\% at the LLC level, and reduces miss
bandwidth by up to 84\% at the L1 and 92\% at the LLC. Correspondingly
reduces on-chip memory hierarchy energy by as much as 36\% and
improves performance by as much as 50\%.
 %% abstract
\input{files/02-dedquot} %% dedication and quotation, if any 
%!TEX root=/home/ska124/Dropbox/Thesis/thes-full.tex
%% Copyright 1998 Pepe Kubon
%%
%% `ack.tex' --- aknowledgments for thes-full.tex, thes-short-tex from
%%               the `csthesis' bundle
%%
%% You are allowed to distribute this file together with all files
%% mentioned in READ.ME.
%%
%% You are not allowed to modify its contents.
%%

%%%%%%%%%%%%%%%%%%%%%%%%%%%%%%%%%%%%%%%%%%%
%
%       Acknowledgment 
%
%%%%%%%%%%%%%%%%%%%%%%%%%%%%%%%%%%%%%%%%%%

\prefacesection{Acknowledgments}

I would like to thank Binay Meso and Supriya Masi; without them I would not be here today, Dr. Arrvindh Shriraman; for his guidance and being a constant source of motivation, Hongzhou Zhao and  Dr. Sandhya Dwarkadas; for all the help and advice while working on the Amoeba Cache projects, and finally my friends and colleagues for their help and support.
























 %%  acknowledgments

%%%  generate contents, lists of figures, etc.
\lists

%% preface (foreword), if any
\input{files/04-preface} 

%%% prepare main section
\beforetext

%%% main matter - chapters
%!TEX root=/home/ska124/Dropbox/Thesis/thes-full.tex
%% Copyright 1998 Pepe Kubon
%%
%% `one.tex' --- 1st chapter for thes-full.tex, thes-short-tex from
%%                the `csthesis' bundle

%%%%%%%%%%%%%%%%%%%%%%%%%%%%%%%%%%%%%%%%%%%%%%%%%
%
%       Chapter 1 
%
%%%%%%%%%%%%%%%%%%%%%%%%%%%%%%%%%%%%%%%%%%%%%%%%

\chapter{Introduction}
\label{introduction}

Memory systems are an integral part of computer architecture whose overall design and organisation have remained unchanged since their inception. Early mainframe computers in the 1960's were known to use a hierarchial memory organisation. The memory technologies include semi-conductor, magnetic core, drum and disc. Caching would be used to fetch data and instructions into the fastest memory ahead of CPU accesses. Initally, accessing memory was only slightly slower than register access however as the difference grew, the need to mitigate the delay incurred for a memory access became extremely important. The rate at which computations were performed kept increasing however the rate at which data was fed to the processor from the memory system did not grow at the same rate. In order to alleviate the effects of slow memory, smaller faster memory was built close to the processor to cache frquently used data. The first documented use of a data cache was in the IBM System/360 Model 85\cite{liptay68}. Now multilevel memory hierarchies are used which are composed of fast static random access memory and slower dynamic random access memory before going to disk.


\section{Cache Memory Systems}

Long story about cache memory systems and how they are organised
Diagrams of standard caches

\section{Motivation for Change}

In traditional caches, the cache block defines the fundamental unit of
data movement and space allocation in caches. The blocks in the data
array are uniformly sized to simplify the insertion/removal of blocks,
simplify cache refill requests, and support low complexity tag
organization. Unfortunately, conventional caches are inflexible (fixed
block granularity and fixed \# of blocks) and caching efficiency is
poor for applications that lack high spatial locality.  Cache blocks
influence multiple system metrics including bandwidth, miss rate, and
cache utilization. The block granularity plays a key role in
exploiting spatial locality by effectively prefetching neighboring
words all at once. However, the neighboring words could go unused due
to the low lifespan of a cache block. The unused words occupy
interconnect bandwidth and pollute the cache, which increases the \#
of misses. We evaluate the influence of a fixed
granularity block below.

\subsection{Cache Utilization}

In the absence of spatial locality, multi-word cache blocks (typically 64
bytes on existing processors) tend to increase cache pollution and fill the
cache with words unlikely to be used.  To quantify this pollution, we segment
the cache line into words (8 bytes) and track the words touched before the
block is evicted.  We define utilization as the average \# of words touched in
a cache block before it is evicted. We study a comprehensive collection of
workloads from a variety of domains: 6 from PARSEC~\cite{Bienia:2008:PBS:1454115.1454128}, 7 from
SPEC2006, 2 from SPEC2000, 3 Java workloads from DaCapo~\cite{Blackburn:2006:DBJ:1167473.1167488}, 3
commercial workloads (Apache, SpecJBB2005, and TPC-C~\cite{Llanos:2006:TOT:1228268.1228270}), and the
Firefox web browser.  Subsets within benchmark suites were chosen based on
demonstrated miss rates on the fixed granularity cache (i.e., whose working
sets did not fit in the cache size evaluated) and with a spread and diversity
in cache utilization.  We classify the benchmarks into 3 groups
based on the utilization they exhibit: Low ($<$33\%), Moderate (33\%---66\%),
and High (66\%+) utilization (see Table~\ref{table:benchmark_categories}).

%!TEX root=/home/ska124/Dropbox/Thesis/thes-full.tex
\begin{table}[t]
\centering
  \begin{tabular}{|c|c|p{0.65\textwidth}|}
    \hline
    Group & Utilization \% & ~~Benchmarks \\
    \hline
    Low        & 0 --- 33\% & art, soplex, twolf, mcf, canneal, lbm, omnetpp \\
    \hline
    Moderate   & 34 --- 66\% & astar, h2, jbb, apache, x264, firefox, tpc-c, freqmine, fluidanimate \\
    \hline
    High       & 67 --- 100\% & tradesoap, facesim, eclipse, cactus, milc, ferret \\
    \hline
  \end{tabular}

\caption{Benchmark Groups}
\label{table:benchmark_categories}
\end{table}


\begin{figure}[!h]

 \begin{center}
  \includegraphics[width=0.5\textwidth]{files/Plots/05-StackBar_Word_Access_64K.pdf}
  \caption{Distribution of words touched in a cache
    block. Avg. utilization is on top. (Config:
    64K, 4 way, 64-byte block.)}
  \label{fig:stackbar_words_64k}
 \end{center}

\end{figure}

Figure~\ref{fig:stackbar_words_64k} shows the histogram of words
touched at the time of eviction in a cache line of a 64K, 4-way cache 
(64-byte block, 8 words per block) across the different benchmarks. 
Seven applications have
less than 33\% utilization and 12 of them are dominated (>50\%) by 1-2
word accesses.  In applications with good spatial locality (cactus,
ferret, tradesoap, milc, eclipse) more than 50\% of the evicted blocks have
7-8 words touched. Despite similar average utilization for
applications such as astar and h2 (39\%), their distributions
are dissimilar; $\simeq$70\% of the blocks in astar have 1-2 words
accessed at the time of eviction, 
whereas $\simeq$50\% of the blocks in h2 have 1-2 words accessed per block.  
Utilization for a single application also changes over time; for example, ferret's 
average utilization, measured as the average fraction of words used in 
evicted cache lines over 50 million instruction windows, 
varies from 50\% to 95\% with a periodicity of roughly 400 million instructions.   
%Different temporal phases and 
%memory regions
%within an application may have different requirements: in firefox, 40\% of
%the blocks have 1--2 words touched, 26\% of the blocks have 3--4 words
%touched, and 34\% have 5+ words touched. 
%Overall, the utilization plot
%indicates the need for run time adjustment of cache block granularity.


\subsection{Effect of Block Granularity on Miss Rate and Bandwidth}

Cache miss rate directly correlates with performance, while under
current and future wire-limited technologies, bandwidth
directly correlates with dynamic energy.
Figure~\ref{fig:scatter_bw_64k_1m} shows the influence of block
granularity on miss rate and bandwidth for a 64K L1 cache and a 1M L2
cache keeping the number of ways constant. For the 64K L1, the plots
highlight the pitfalls of simply decreasing the block size to
accommodate the Low group of applications; miss rate increases by
2$\times$ for the High group when the block size is changed from 64B
to 32B; it increases by 30\% for the Moderate group. A smaller block
size decreases bandwidth proportionately but increases miss rate. With
a 1M L2 cache, the lifetime of the cache lines increases significantly, 
improving overall utilization. Increasing the block size from
64$\to$256 halves the miss rate for all application groups. 
The bandwidth is increased by 2$\times$ for the Low and Moderate.

% SK- Apache 3x miss rate drop not observed anymore. Occured due to
% different length warmup for Oracle and Fixed runs.

Since miss rate and bandwidth have different optimal block
granularities, we use the following metric: $\frac{1}{Miss Rate \times
  Bandwidth}$ to determine a fixed block granularity suited to an
application that takes both criteria into account.
Table~\ref{table:bwmr_classify} shows the block size that maximizes
the metric for each application.  It can be seen that different
applications have different block granularity requirements.  For
example, the metric is maximized for apache at 128 bytes and for
firefox (similar utilization) at 32 bytes.  Furthermore, the optimal
block sizes vary with the cache size as the cache lifespan
changes. This highlights the challenge of picking a single block size
at design time especially when the working set does not fit in the
cache.
%In short, \textit{the tradeoff between increasing the 
%  cache block granularity to achieve spatial prefetching and lower miss rate, 
%  and reducing the granularity to minimize traffic and pollution needs
%  adaptive cache blocks.}

\subsection{Need for adaptive cache blocks}
Our observations motivate the need for adaptive cache line
granularities that match the spatial locality of the data access patterns
in an application. In summary:
\begin{itemize}
  \item  Smaller cache lines improve utilization but tend to increase
    miss rate and potentially traffic for applications with good
    spatial locality, affecting the overall performance.
  \item Large cache lines pollute the cache space and interconnect
    with unused words for applications with poor spatial locality, 
   significantly decreasing the caching efficiency.
\item Many applications waste a significant fraction of the cache
  space. Spatial locality varies not only across applications but also
  within each application, for different data structures as well as 
  different phases of access over time.   
%\item Optimizing miss rate (impacts performance), bandwidth (impacts
%  dynamic energy), and utilization simultaneously with a fixed size
%  cache line is not tractable, since in many cases the optimality is
%  achieved at different block sizes.
\end{itemize}


\begin{figure}[!t]

  \subfloat[64K - Low]{
    \includegraphics[width=0.24\textwidth]{files/Plots/05-Scatter_Bw_Miss_64K_low.pdf}
  }
  \subfloat[1M - Low]{
     \includegraphics[width=0.24\textwidth]{files/Plots/05-Scatter_Bw_Miss_1M_low.pdf}
  }
  
  \subfloat[64K - Moderate]{
    \includegraphics[width=0.24\textwidth]{files/Plots/05-Scatter_Bw_Miss_64K_mod.pdf}
  }
  \subfloat[1M - Moderate]{
     \includegraphics[width=0.24\textwidth]{files/Plots/05-Scatter_Bw_Miss_1M_mod.pdf}
  }
    
  \subfloat[64K - High]{
    \includegraphics[width=0.24\textwidth]{files/Plots/05-Scatter_Bw_Miss_64K_high.pdf}
  }
  \subfloat[1M - High]{
     \includegraphics[width=0.24\textwidth]{files/Plots/05-Scatter_Bw_Miss_1M_high.pdf}
  }

  \caption{Bandwidth vs. Miss Rate. (a),(c),(e): 64K, 4-way
    L1. (b),(d),(f): 1M, 8-way LLC.  Markers on the plot indicate cache
    block size. Note the different scales for different groups.}
  \label{fig:scatter_bw_64k_1m}
  \end{figure}


\begin{table}[!h]
\caption[Optimal block size]{Optimal block size. Metric: $\mathbf{\frac{1}{Miss-rate \times Bandwidth}}$}
\label{table:bwmr_classify}
\begin{center}
{
\small
  \begin{tabular}{ |@{~}c@{~}| m{0.72\columnwidth} |}
    \hline
    \multicolumn{2}{|c|}{64K, 4-way} \\
    \hline
    Block  & Benchmarks \\
    \hline
    32B   & cactus, eclipse, facesim, ferret, firefox, fluidanimate,freqmine, milc, tpc-c, tradesoap \\
    \hline
    64B   &  art \\
    \hline
    128B  & apache, astar, canneal, h2, jbb, lbm, mcf, omnetpp, soplex, twolf, x264 \\
    \hline
    \multicolumn{2}{|c|}{1M, 8-way} \\
    \hline
    Block & Benchmarks \\
    \hline
    64B  & apache, astar, cactus, eclipse, facesim, ferret, firefox, freqmine, h2, lbm, milc, omnetpp, tradesoap, x264\\
    \hline
    128B & art\\
    \hline
    256B  & canneal, fluidanimate, jbb, mcf, soplex, tpc-c, twolf\\
    \hline
  \end{tabular}
}
\end{center}
\end{table}




\section{Dissertation Outline} 
%!TEX root=/home/ska124/Dropbox/Thesis/thes-full.tex
%%%%%%%%%%%%%%%%%%%%%%%%%%%%%%%%%%%%%%%%%%%%%%%%%%%%
%
%     Chapter 3   
%
%%%%%%%%%%%%%%%%%%%%%%%%%%%%%%%%%%%%%%%%%%%%%%%%%%%

\chapter{Amoeba Cache Architecture}
\label{chap:ac_architecture}

As described in Section \ref{sec:cache_memory_systems}, a conventional cache organises the data array into a 2 dimensional structure. A transparently addressed cache uses the same namespace (memory address space layout) as the main memory. The blocks which are stored in the sets are \textit{tagged} with the aligned start address of block present in the main memory. The \textit{tags} for the cache blocks currently present in the cache set are maintained in a separate array. When a search is being performed to find out whether a required physical address is present in the cache, the tag array is looked up to determine a cache hit or a cache miss. The organization of the cache set and tag array is shown in Figure \ref{fig:set_assoc_arch}. The effective address is the virtual address supplied by the CPU of the required datum. The component bits of the effective address is segmented into 3 parts which form the \textit{Virtual Page Number(VPN)}, set number and byte offset. The set number and byte offset are looked up in the tag array while the VPN is looked up in the \textit{Translation Lookaside Buffer(TLB)} to check that the current process has brought in the corresponding page and it is valid. The organisation described (and shown in Fig \ref{fig:set_assoc_arch}) is virtually indexed, physically tagged organisation where the lookup logic does not include the TLB translation in the critical path to enable faster searches. There are other organisations such as virtually indexed, virtually tagged and physically indexed, physically tagged which are uncommon due to inherent issues with their design. The tradeoff for a virtually indexed, physically tagged cache is that it can only grow in size with an increase in the associativity of each set, or an increase in the size of each cache block. The Intel Sandy Bridge architecture is known to use a virtually indexed, physically tagged cache organisation.



\begin{figure}[b]
  %% Pg 84 - Jacob
  \begin{center}
    \includegraphics[width=\textwidth]{files/Figures/06-NWaySetAssocCache.pdf}
    \caption[Conventional N-Way Set-Associative Cache]{\textbf{Conventional N-Way Set-Associative Cache} \bigspot{1} The Virtual Page Number (VPN) is used to look up the entry in the Translation Lookaside Buffer (TLB) \bigspot{2} According to the number of sets in the cache, the following bits from the address are used to look up the corresponding set from the cache \bigspot{3} The tags read out from the set are compared with the translation from the TLB and tested for equality \bigspot{4} The corresponding cache block is forwarded to the output buffer for the tag which matches the TLB lookup \bigspot{5} Using the byte offset from the CPU, the mutiplexer selects the correponding critical word }
    \label{fig:set_assoc_arch}
  \end{center}
\end{figure}

\clearpage

In contrast to a conventional cache, the \AC\ architecture enables the memory hierarchy to fetch and allocate space for a range of words (i.e. a variable granularity cache block) based on the spatial locality of the application. For example, consider a 64K cache (256 sets) that allocates 256 bytes per set. These 256 bytes can adapt to support, for example, eight 32-bytes blocks, thirty-two 8-byte blocks, or four 32-byte blocks and sixteen 8-byte blocks, based on the set of contiguous words likely to be accessed. 
\\ \\
The key challenges to realising the \AC\ architecture are
\begin{enumerate}[noitemsep]
	\item To support a variable number of blocks per set
	\item To support a variable granularity for each block
	\item To support a variable number of tags, which correspond to the blocks in the set
\end{enumerate}

\begin{figure}[h]
  %% Pg 84 - Jacob
  \begin{center}
    \includegraphics[width=\textwidth]{files/Figures/06-AmoebaCacheArch.pdf}
    \caption[Amoeba Cache Overview]{\textbf{Amoeba Cache Overview} The static RAM (SRAM) array where the tags and data are colocated is shown on the right. The \code{T? Bitmap} and the \code{V? Bitmap} for the \AC\ are shown on the left. Each block in the SRAM array represents 8 bytes (1 word). In this specific example, we show an \AC\ with 4 sets and 1024 bytes per set. The invalid, data and tag words (marked in the SRAM array) are tracked by setting the corresponding bits in the \code{T? and V? Bitmaps}. This information is maintained in order to simplify cache operations such as insertion and refill. }
    \label{fig:amoeba_cache_arch}
  \end{center}
\end{figure}

The \AC\ adopts a solution inspired by software data structures, where programs hold meta-data and actual data entries in the same address space. To achieve maximum flexibility, the \AC\ completely eliminates the tag array and collocates the tags with the actual data blocks (see Figure~\ref{fig:amoeba_cache_arch}). To distinguish which words are data words and which ones are tags within the set, we use a bitmap data structure (labeled \code{T? Bitmap} in Fig~\ref{fig:amoeba_cache_arch}). For each word in the set which is a tag, we set the corresponding bit in the \code{T? Bitmap}. We also decouple the conventional valid/invalid bits (typically associated with the tags) and organize them into a separate array (labeled \code{V? Bitmap} in Fig~\ref{fig:amoeba_cache_arch}) to simplify block replacement and insertion. \AC\ tags are composed of a \code{Region Tag} and a tuple which consists of the \code{Start} and \code{End} address of the variable granularity cache block. The data block immediately follows the tag word as shown in Fig~\ref{fig:amoeba_cache_arch}. The following sections provide more detail about the \AC\ architecture and how cache operations are performed.


\section{Amoeba Blocks and Set-Indexing}


\begin{figure}[h]
  %% Get images from http://en.wikipedia.org/wiki/CPU_cache#Associativity
  \subfloat[Memory Regions]{
    \includegraphics[width=0.5\textwidth]{files/Figures/06-MemoryRegions.pdf}
  }
  \subfloat[Addressing]{
     \includegraphics[width=0.5\textwidth]{files/Figures/06-Addressing.pdf}
  }
  \caption[Memory Regions]{ (a) The linear memory address space is segmented into Regions. The \AB{}s are constrainted to have their start and end within a single memory region. (b) 64 bits are used to encode the Region Tag, the Set Index, the start or end word and the word offset in the tag for an \AB{}.  }
  \label{fig:mem_region_addr}
\end{figure}


The \AC\ data array holds a collection of varied granularity \AB{}s that do not overlap. Each \AB\ is a 4 tuple consisting of \code{\textless RegionTag, Start, End, Data-Block\textgreater} (Figure~\ref{fig:amoeba_cache_arch}). The first 3 components of the tuple are equivalent to a tag in a conventional cache. We allocate 8 bytes (1 word) for each tag. In order to simplify cache lookups for \AB{}s, we partition the address space into \code{Regions}. A \code{Region} is an aligned block of memory of size \code{RMAX} bytes. The boundaries of any \AB{} block (\code{Start} and \code{End}) are constrained to lie within the regions' boundaries. The minimum granularity of the data in an \AB\ is 1 word and the maximum is \code{RMAX} words. We can encode \code{Start} and \code{End} in $log_2(RMAX)$ bits. The set indexing function masks the lower $log_2(RMAX)$ bits to ensure that all \AB{}s (every memory word) from a region index to the same set. The Region Tag and Set-Index are identical for every word in the \AB{}. Retaining the notion of \textit{sets} enables fast lookups and helps elude challenges such as synonyms (same memory word mapping to different sets). When comparing against a conventional cache, we set \code{RMAX} to 8 words (64 bytes), ensuring that the set indexing function is identical to that in the conventional cache to allow for a fair evaluation.



\section{Data Lookup}

When data is referenced by the CPU, a cache lookup takes place in order to determine whether the required datum is present in the cache or not (resulting in a cache hit or miss). The operation percolates down the memory hierarchy until a cache returns a hit or the backing store supplies the datum required. Fig~\ref{fig:set_assoc_arch} shows a conventional cache which operates in \textit{Fast Mode}, where the contents of the entire set is read out into the output buffer in parallel with the tag lookup. Megabyte sized caches, with larger sets, may want to avoid the extra cost of reading out all ways to the output buffer and wait until the tag lookup completes to read out only the correct way from the set. Though this saves energy, it serializes the lookup and takes longer. The delay is usually tolerated as the \textit{Serial Mode} lookup is often implemented in the L2 caches or lower in the memory heirarchy. Another approach which minimises energy whilst still reading out the way in parallel is \textit{way prediction}\cite{patent:DataCacheWayPrediction,patent:WayPredictionVirtualHint}, commonly used in processors manufactured by MIPS.
\\

In contrast to a conventional cache, the \AC\ needs to lookup tags from the SRAM array to determine a hit or miss. The metadata stored in the \code{T? Bitmap} is used during the lookup operation in the \AC{}. Figure~\ref{fig:ac_lookup} describes the steps of the lookup procedure in an \AC{}. The overheads incurred due to the extra stages introduced in the critical path are evaluated in chapter \ref{sec:hardware_complexity}.


\begin{figure}[h]
  \includegraphics[width=\textwidth]{files/Figures/06-Lookup.pdf}
  \caption[Amoeba Cache Lookup]{\textbf{Amoeba Cache Lookup} : The incoming effective address is segmented into the region tag, set index and word offset. \bigspot{1} The \code{Tag? Bitmap} is looked up to determine which words in the activated set are required for the tag comparison. Note that given the minimum size of a \AB{} is two words (1 word for the tag metadata, 1 word for the data), adjacent words cannot be tags. Given this constraint, 
  the number of 2-1 multiplexers required to route one of the adjacent words to the comparator ($\in$ operator), is equal to half the number of words in the set. \bigspot{2} Simultaneously, the set is activated and the contents are latched onto the output buffer. \bigspot{3} The appropriate tag words are selected with the input from the the \code{Tag? Bitmap}. \bigspot{4} The comparator generates the hit signal for the word selector. The $\in$ operator consists of two comparators: a) an aligned \code{Region tag} comparator, a conventional $==$ (64 - $log_2N_{sets}$ - $log_2{RMAX}$ bits wide, e.g., 50 bits) that checks if the \AB{} belongs to the same region and b) a $ Start <= W < END $ range comparator ($log_2RMAX$ bits wide; e.g., 3 bits) that checks if the \AB\ holds the required word. Finally, in \bigspot{5}, the tag match activates and selects the appropriate word. The critical path (as indicated on the left) includes the read out from the set, the tag selectors and the $\in$ operation.}
  \label{fig:ac_lookup}
\end{figure}

\clearpage

\section{Block Insertion}

On a miss for the desired word, a spatial granularity predictor is invoked (see Section~\ref{sec:spatial_pattern_predictor}), which specifies the range of the \AB{} to refill. To determine a position in the set to slot the incoming block we can leverage the \code{Valid? Bitmap}. The \code{V? Bitmap} has 1 bit/word in the cache; a ``1'' bit indicates the word has been allocated (valid data or a tag). To find space for an incoming data block we perform a substring search on the \code{V? Bitmap} of the cache set for contiguous 0s (empty words). For example, to insert an \AB{} of five words (four words for data and one word for tag), we perform a substring search for \textit{00000} in the \code{V? Bitmap} / set (e.g., 32 bits wide for a 64K cache). If we cannot find the space, we keep triggering the replacement algorithm until we create the requisite contiguous space. Following this, the \AB\ tuple (Tag and Data block) is inserted, and the corresponding bits in the \code{T? and V? Bitmaps} are set. The 0s substring search can be accomplished with a lookup table; many current processors already include a substring instruction. Intel SSE 4.2 include the instruction, \code{PCMPISTRI}, which can accomplish the required task.

\section{Replacement Policy}

The replacement policy of a cache determines which blocks are evicted when new data is brought in and there is no room to store it. The replacement algorithm tries to make an optimal choice where it evicts a blocks which is not expected to be used in the near future. The most optimal choice possible would be to remove a block which is not going to be referenced in the program again or which is going to be referenced farthest in the future in comparison to the other cache blocks. The optimal algorithm is also known as \textit{"Belady's Optimal Algorithm"}, named after Hungarian computer scientist, Laszlo Belady.
\\

The Least Recently Used (LRU) algorithm is a popular choice for conventional caches and is based on the principle of temporal locality of reference. The hardware tracks data reference to the different cache blocks and when a new insertion request arrives, the least recently used block in the set is evicted. Implementing an LRU cache in software is relatively easy with the use of a linked list however, in hardware requires a large amount of resources. Thus hardware manufactureres commonly implement the \textit{Pseudo-LRU}(PLRU) which has a lower metadata overhead and is a reasonable approximation of LRU. The tree based PLRU was implemented in processors such as the Intel 80486 and many processors in the PowerPC family. 

\subsection{\AC\ Replacement Policy}

To reclaim the space from an \AB\, the tag bits T? (tag) and V? (valid) bits corresponding to the block are unset. The key issue is identifying the \AB\ to replace. Classical pseudo-LRU algorithms~\cite{Kedzierski-ipdps-2010,manual:OpenSparcT1} keep the metadata for the replacement algorithm separate from the tags to reduce port contention. To be compatible with pseudo-LRU and other algorithms such as DIP~\cite{qureshi-isca-2007} that work with a fixed \# of ways, we can logically partition a set in \AC\ into $N_{ways}$. For instance, if we partition a 32 word cache set into 4 logical ways, any access to an \AB\ tag found in words 0 ---7 of the set is treated as an access to logical way 0. Finding a replacement candidate involves identifying the selected replacement way and then picking (possibly randomly) a candidate \AB{}. 

\subsection{Other Replacement Policies}

More refined replacement algorithms that require per-tag metadata can harvest the space in the tag-word of the \AB\, which is 64 bits wide (for alignment purposes) while physical addresses rarely extend beyond 48 bits.


% \subsection{Partial Misses} \label{sec:partialmiss} 

% With variable granularity data blocks, a challenging although rare case (5
% in every 1K accesses) that occurs is a \textit{partial miss}. It is observed
% primarily when the spatial locality changes.  Figure~\ref{fig:partial-miss}
% shows an example. Initially, the set contains two blocks from a region R, one
% \AB\ caches words 1 --3 (Region:R, START:1 END:3) and the other holds words 5
% --6 (Region:R START:5 END:6). The CPU reads word 4, which
% misses, and the spatial predictor requests an \AB\ with range START:0 and
% END:7. The cache has \AB{}s that hold subparts of the incoming \AB{}, and some
% words (0, 4, and 7) need to be fetched.

% \AC\ removes the overlapping sub-blocks and allocates a new \AB{}. This is a
% multiple step process:\spot{1} On a miss, the cache identifies the overlapping
% sub-blocks in the cache using the tags read out during lookup. $\cap \neq
% NULL$ is true if $START_{new}$ < $END_{Tag}$ and $END_{new}$ > $START_{Tag}$
% ($New=$ incoming block and $Tag=$ \AB\ in set).  \spot{2} The data blocks that
% overlap with the miss range are evicted and moved one-at-a-time to the MSHR
% entry. \spot{3} Space is then allocated for the new block, i.e.,
% it is treated like a new insertion. \spot{4} A miss request is issued for the
% entire block (START:0 --- END:7) even if only some words (e.g., 0, 4, and 7)
% may be needed. This ensures request processing is simple and only a single
% refill request is sent.\spot{5} Finally, the incoming data block is patched
% into the MSHR; only the words not obtained from the L1 are copied (since the
% lower level could be stale).


% \begin{figure}[!h]
% \centering
% \includegraphics[width=0.37\textwidth]{figures/PartialMiss}
% \begin{minipage}[b]{0.5\textwidth} 
% { 
% \centering \spot{1}
%     Identify blocks overlapping with \code{New} block. \spot{2}
%     Evict overlapping blocks to MSHR. \spot{3} Allocate space
%     for new block (treat it like a new insertion). \spot{4} Issue
%     refill request to lower level for entire block. \spot{5} Patch
%     only newer words as lower-level data could be stale.  
% %(e.g., writeback cache)
% }
% \end{minipage}
% \caption{Partial Miss Handling. Upper: Identify relevant
%   sub-blocks. Useful for other cache controller events as well, e.g.,
%   recalls. Lower: Refill of words and insertion.}
% \label{fig:partial-miss}
% \end{figure}

\section{Spatial Pattern Predictor}
\label{sec:spatial_pattern_predictor}

\section{Related Work}
\subsection{Line Distillation}
\subsection{Sector Caches}
\subsection{Indexed Indirect Caches}
\subsection{Cache Compression}

\input{files/07-implementation}
%!TEX root=/home/ska124/Dropbox/Thesis/thes-full.tex

%%%%%%%%%%%%%%%%%%%%%%%%%%%%%%%%%%%%%%%%%%%%%%%%%
%
%     Chapter 5
%
%%%%%%%%%%%%%%%%%%%%%%%%%%%%%%%%%%%%%%%%%%%%%%%%

\chapter{Evaluation}
\label{chap:evaluation}
This chapter evaluates the performance of the \AC\ described in Chapter~\ref{chap:ac_architecture}. The evaluation is performed using the infrastructure described in Chapter~\ref{sec:simulation_infrastructure}. This chapter is divided into sections based on performance evaluations of:
\begin{itemize}[noitemsep]
	\item Comparison against a fixed granularity cache
	\item Adaptivity of the \AC\
	\item The spatial pattern predictor
	\item \AC\ versus other approaches
\end{itemize}

\section{Improved Memory Hierarchy Efficiency}
\label{sec:efficiency}
\vspace{5pt}
\noindent \textbf{Result 1:}{\emph~{\AC\ increases cache capacity by harvesting
  space from unused words and can achieve an 18\% reduction in both L1 and
  L2 miss rate.}
\\ \\
\noindent \textbf{Result 2:}{\emph~{\AC\ adaptively sizes the cache block
  granularity and reduces L1$\leftrightarrow$L2 bandwidth by 46\% and
  L2$\leftrightarrow$Memory bandwidth by 38\%.}
}
\\ \\
In this section, the bandwidth and miss rate properties of an \AC\ are compared against a conventional cache. A \textit{Fixed} cache represents a conventional cache which allocates a fixed granularity cache block on a refill request. The accuracy of the spatial pattern predictor is an important factor which governs the accuracy of the \AC\ and is evaluated separately. For the results presented in this section, cache line utilisation statistics, gathered from a prior run of the application on a conventional cache, are used to drive the predictor. This isolates the benefits of the \AC\ from the potentially changing accuracy of the spatial pattern predictor across different cache geometries. This also ensures that the spatial granularity predictions can be replayed across multiple simulation runs. To ensure equivalent data storage space, the \AC\ size is set to the sum of the tag array and the data array in a conventional cache. At the L1 level (64K), the net capacity of the \AC\ is 64K + 8*4*256 bytes and at the L2 level (1M) configuration, it is 1M + 8*8*2048 bytes. The L1 cache has 256 sets and the L2 cache has 2048 sets. 

Fig~\ref{fig:eval_scatter_bw_64k_1m} plots the miss rate and the traffic characteristics of the \AC{}.  Since \AC{} can hold blocks varying in size from 8B to 64B, each set can hold more blocks by utilizing the space saved from eliminating untouched words. The \AC\ reduces the 64K L1 miss rate on average by 23\%\footnote{All reported averages are geometric mean unless otherwise specified.} and standard deviation(SD) of 24 for the Low group, and by 21\%(SD:16) for the moderate group; even applications with high spatial locality experience a 7\%(SD:8) improvement in miss rate. There is a 46\%(SD:20) reduction on average in L1$\leftrightarrow$L2 bandwidth. At the 1M L2 level, the \AC\ improves the moderate group's miss rate by 8\%(SD:10) and bandwidth by 23\%(SD:12).  Applications with moderate utilization make better use of the space harvested from unused words by \AC{}. Many low utilization applications tend to be streaming and providing extra cache space does not help lower miss rate. However, by not fetching unused words, the \AC\ achieves a significant reduction (38\%(SD:24) on average) in off-chip L2$\leftrightarrow$Memory bandwidth; even High utilization applications see a 17\%(SD:15) reduction in bandwidth.  Utilization and miss rate are not, however, always directly correlated (more details in \S~\ref{sec:adaptivity}).

With the \AC\, the number of blocks per set varies based on the granularity of the blocks being fetched, which in turn depends on the spatial locality in the application. Table~\ref{table:blockcount} shows the average number of blocks per set. In applications with low spatial locality, the \AC\ adjusts the block size and adapts to store many smaller blocks. The 64K L1 \AC\ stores 10 blocks per set for mcf and 12 blocks per set for art, effectively increasing associativity without introducing hardware overheads. At the L2, when the working set starts to fit in the L2 cache, the set is partitioned into fewer blocks. 

\begin{figure}[ht]

  \subfloat[64K - Low]{
    \includegraphics[width=0.48\textwidth]{files/Plots/08-Scatter_Bw_Miss_64K_low.pdf}
  }
  \subfloat[1M - Low]{
     \includegraphics[width=0.48\textwidth]{files/Plots/08-Scatter_Bw_Miss_1M_low.pdf}
  }
  
  \subfloat[64K - Moderate]{
    \includegraphics[width=0.48\textwidth]{files/Plots/08-Scatter_Bw_Miss_64K_mod.pdf}
  }
  \subfloat[1M - Moderate]{
     \includegraphics[width=0.48\textwidth]{files/Plots/08-Scatter_Bw_Miss_1M_mod.pdf}
  }
    
  \subfloat[64K - High]{
    \includegraphics[width=0.48\textwidth]{files/Plots/08-Scatter_Bw_Miss_64K_high.pdf}
  }
  \subfloat[1M - High]{
     \includegraphics[width=0.48\textwidth]{files/Plots/08-Scatter_Bw_Miss_1M_high.pdf}
  }

  \caption[Bandwidth vs. Miss Rate]{Bandwidth vs. Miss Rate for a fixed granularity cache and \AC{}. (a),(c),(e): 64K, 4-wayL1 equivalent (b),(d),(f): 1M, 8-way LLC equivalent.  Markers on the plot indicate cache block size. Note the different scales for different groups.}
  \label{fig:eval_scatter_bw_64k_1m}
\end{figure}

\clearpage

%!TEX root=/home/ska124/Dropbox/Thesis/thes-full.tex
\begin{table}[h]
  \centering
  \begin{tabular}{|c|c|}
  \hline 
    Blocks/Set & 64K Cache, 288 B/set \\
    \hline
    4---5  &  ferret, cactus, firefox, eclipse, facesim, freqmine, milc, astar\\
    6---7  &  tpc-c, tradesoap, soplex, apache, fluidanimate\\
    8---9  &  h2, canneal, omnetpp, twolf, x264, lbm, jbb \\
    10---12 & mcf, art \\
    \hline
    \multicolumn{2}{c}{} \\
    \hline
    Blocks/Set & 1M Cache, 576 B/set \\
    \hline
    3---5 & eclipse, omnetpp   \\
    8---9 & cactus, firefox, tradesoap, freqmine, h2, x264, tpc-c   \\
    10---11  & facesim, soplex, astar, milc, apache, ferret  \\
    12---13  & twolf, art, jbb, lbm , fluidanimate \\
    15---18 & canneal, mcf \\
    \hline
  \end{tabular}
  \caption[Amoeba Blocks per set]{Average number of \AB{}s per set}
  \label{table:blockcount}
\end{table}


Note that applications like eclipse and omnetpp hold only 3---5 blocks per set on average (lower than conventional associativity) due to their low miss rates (see Table~\ref{table:Abs_Eval_Oracle}). With streaming applications (e.g., canneal), \AC\ increases the number of blocks/set to $>$15 on average. Finally, some applications like apache store between 6---7 blocks/set with a 64K cache with varied block sizes (see Figure~\ref{fig:StackBar_PredictorSize}): approximately 50\% of the blocks store 1-2 words and 30\% of the blocks store 8 words at the L1. As the size of the cache increases and thereby the lifetime of the blocks, the \AC\ adapts to store larger size blocks as can be seen in Figure~\ref{fig:StackBar_PredictorSize}.
\\ \\
\indent Utilization is improved greatly across all applications (90\%+ in many cases). Figure~\ref{fig:StackBar_PredictorSize} shows the distribution of cache block granularities in \AC{}. The \AB\ distribution matches the word access distribution presented in Fig~\ref{fig:stackbar_words_64k}). With the 1M cache, the larger cache size improves block lifespan and thereby utilization, with a significant drop in the \% of 1---2 word blocks. However, in many applications (tpc-c, apache, firefox, twolf, lbm, mcf), up to 20\% of the blocks are 3--6 words wide, indicating the benefits of adaptivity and the challenges faced by a fixed granularity conventional cache.

\begin{figure}[!h]
  \centering
  \subfloat[64K L1 cache]{
    \includegraphics[width=0.8\textwidth]{files/Plots/08-StackBar_PredictSize_64K.pdf}
  }

  \subfloat[1M L2 cache]{
    \includegraphics[width=0.8\textwidth]{files/Plots/08-StackBar_PredictSize_1M.pdf}
  }
  \caption[Distribution of cache block sizes]{Distribution of cache line granularities in the (a) 64K L1 and (b) 1M L2 \AC{}. Average utilization is on top.}
  \label{fig:StackBar_PredictorSize}
\end{figure}

\clearpage

\begin{figure}[!h]
  \centering
  \subfloat[64K L1 cache]{
    \includegraphics[width=\textwidth]{files/Plots/08-Oracle-64K-Miss-BW-Improvement.pdf}
  }

  \subfloat[1M L2 cache]{
    \includegraphics[width=\textwidth]{files/Plots/08-Oracle-1M-Miss-BW-Improvement.pdf}
  }
  \caption[Miss Rate and Bandwidth Improvement]{Miss Rate and bandwidth reduction with respect to a fixed granularity conventional cache with cache line size 64 bytes for (a) 64K equivalent \AC\ and (b) 1M equivalent \AC{}.}
  \label{fig:miss_bw_reduction}
\end{figure}

\section{Overall Performance and Energy}
\label{sec:overall_performance_and_energy}
\noindent \textbf{Result 3:}{~\AC\ improves overall cache efficiency and boosts performance by 10\% on commercial applications\footnote{``Commercial'' applications includes Apache, SpecJBB and TPC-C.}, saving up to 11\% of the energy of the on-chip memory hierarchy. Off-chip L2$\leftrightarrow$memory energy sees a mean reduction of 41\% across all workloads. 

\begin{figure}[!h]
  \centering
    \subfloat[Energy Improvement]{
      \includegraphics[width=\textwidth]{files/Plots/08-Oracle_Energy.pdf}
    }
    
    \subfloat[Performance Improvement]{
      \includegraphics[width=\textwidth]{files/Plots/08-Oracle_Perf.pdf}
    }
    \caption{ The above charts (a) show the percentage reduction in energy and (b) the percentage reduction in cycle time for an \AC\ compared to a fixed granularity conventional cache with 64K L1 and 1M LLC. Higher bars indicate better performance. Y-axis terminated to illustrate bars clearly.}
    \label{plot:multi_sys_perf_energy}
\end{figure}

\clearpage

A two-level cache hierarchy is modeled in which the L1 is a 64K cache with 256 sets (3 cycles load-to-use) and the L2 is 1M, 8192 sets (20 cycles). A fixed memory latency of 300 cycles is assumed. It is assumed that the L1 access is the critical pipeline stage and throttle CPU clock by 4\% (an alternative approach is evaluated in the next section).  The total dynamic energy of the \AC\ is calculated using the energy numbers determined in Section~\ref{sec:area_latency_energy_overhead} through a combination of synthesis and CACTI~\cite{Muralimanohar:2007:ONO:1331699.1331704}. 4 fast tags per set at the L1 and 8 fast tags per set at the L2 are used. The penalty for all the extra metadata in the \AC{} is included. The energy for a single L1---L2 transfer (6.8pJ per byte) is derived from~\cite{weti,Muralimanohar:2007:ONO:1331699.1331704}. The interconnect uses full-swing wires at 32nm, 0.6V. 

Figure~\ref{plot:multi_sys_perf_energy} plots the overall improvement in performance and reduction in on-chip memory hierarchy energy (L1 and L2 caches, and L1$\leftrightarrow$L2 interconnect). For applications that have good spatial locality (e.g., tradesoap, milc, facesim, eclipse, and cactus), the \AC\ has minimal impact on miss rate, but provides significant benefit in terms of reduction in bandwidth. This results in on-chip energy reduction: milc's L1$\leftrightarrow$L2 bandwidth reduces by $\simeq$15\%(see Figure~\ref{fig:miss_bw_reduction}(a)) and its on-chip energy reduces by 5\%. Applications that suffer from cache pollution under \textit{Fixed} (apache, jbb, twolf, soplex and art) see gains in performance and energy. Apache's performance improves by 11\% and on-chip energy reduces by 21\%, while SpecJBB's performance improves by 21\% and energy reduces by 9\%. Art gains approximately 50\% in performance.  Streaming applications like mcf access blocks with both low and high utilization. Keeping out the unused words in the under-utilized blocks prevents the well-utilized cache blocks from being evicted; mcf's performance improves by 12\% and on-chip energy by 36\%.

\subsection{Extra cache pipeline stage}
\label{sec:extra_cache_pipeline_stage}

An alternative strategy to accommodate \AC{}'s overheads is to add an extra pipeline stage to the cache access which increases hit latency by 1 cycle. The CPU clock frequency entails no extra penalty compared to a conventional cache. Using such a design, applications in the moderate and low spatial locality group (8 applications), the \AC\ continues to provide a performance benefit between 6---50\%. milc and canneal suffer minimal impact, with a 0.4\% improvement and 3\% slowdown respectively.  Applications in the high spatial locality group (12 applications) suffer an average 15\% slowdown (maximum 22\%) due to the increase in L1 access latency. In these applications, 43\% of the instructions (on average) are memory accesses and a 33\% increase in L1 hit latency imposes a high penalty. Note that all applications continue to retain the energy benefit. The cache hierarchy energy is dominated by the interconnects and the \AC\ provides notable bandwidth reduction. While these results may change for an out-of-order, multiple-issue processor, the evaluation suggests that \AC\ if implemented with the extra pipeline stage is more suited for lower levels in the memory hierarchy than the L1.  

\begin{figure}[h]
  \centering
  \includegraphics[width=\textwidth]{files/Figures/Placeholder_Chart_Wide.pdf}
  \caption{Placeholder for extra pipeline stage performance plot -- all apps}
  \label{fig:extra_cache_pipeline_stage}
\end{figure}

\subsection{Off-chip L2$\leftrightarrow$Memory energy}
The L2's higher cache capacity makes it less susceptible to pollution and provides less opportunity for improving miss rate. In such cases, the \AC\ keeps out unused words and reduces off-chip bandwidth and thereby off-chip energy. We assume that the off-chip DRAM can provide adaptive granularity transfers for \AC{}'s L2 refills as in~\cite{Yoon_Jeong_Erez_2011}. The DRAM model used was presented in a recent study~\cite{exascale} and consumes 0.5nJ per word transferred off-chip. The low spatial locality applications see a dramatic reduction in off-chip energy. For example, twolf sees a 93\% reduction. On commercial workloads the off-chip energy decreases by 31\% and 24\% respectively. Even for applications with high cache utilization, off-chip energy decreases by 15\%.

\begin{figure}[h]
  \centering
  \includegraphics[width=\textwidth]{files/Figures/Placeholder_Chart_Wide.pdf}
  \caption{Placeholder for off chip energy plot -- all apps}
  \label{fig:offchip_energy}
\end{figure}


\section{Spatial Predictor Tradeoffs}
\label{sec:spatial_predictor_tradeoffs}

The effectiveness of spatial pattern prediction is evaluated in this section. In the table-based approach, a pattern history table records spatial patterns from evicted blocks and is accessed using a prediction index. The table-driven approach requires careful consideration of the following: prediction index, predictor table size and training period. The effects are quantified by comparing the predictor against a baseline fixed-granularity cache. A baseline 64K cache is used since it induces enough misses and evictions to highlight the predictor tradeoffs clearly.

\subsection{Predictor Indexing} 

A critical choice with the history-based prediction is the selection of the predictor index. Two types of predictor indexing were explored:
\begin{itemize}[noitemsep]
  \item a program counter based(\code{PC}) approach\cite{chen-hpca-2004} based on the intuition that fields in a data structure are accessed by specific PCs and tend to exhibit similar spatial behavior. The tag includes the PC and the critical word index: $((PC>>3)<<3)+\frac{(addr\%64)}{8}$.
  \item a \textit{Region}-based (\code{REGION}) approach that is based on the intuition that similar data objects tend to be allocated in contiguous  regions of the address space and tend to exhibit similar spatial behavior. 
\end{itemize}

The miss rate and bandwidth properties of both the  \code{PC} (256 entries, fully associative) and \code{REGION} (1024 entries, 4KB region size) predictors were compared. The size of the table used in each predictor was selected as the optimal found by empirical analysis for each predictor type.  For all applications apart from cactus (a high spatial locality application), \code{REGION}-based prediction tends to overfetch and waste bandwidth as compared to \code{PC}-based prediction, which has 27\% less bandwidth consumption on average across all applications. For 17 out of 22 applications, \code{REGION}-based prediction shows 17\%  better MPKI on average (max: 49\% for cactus). For 5 applications (apache, art, mcf, lbm, and omnetpp), \code{PC} has better accuracy when predicting the spatial behavior of cache blocks than \code{REGION} and demonstrates a 24\% improvement in MPKI (max: 68\% for omnetpp).

\subsection{Predictor Table}

The organization and size of the pattern table was studied using the \code{REGION} predictor. The following parameters were evaluated :
\begin{itemize}[noitemsep]
  \item region size, which directly correlates with the coverage of a fixed-size table
  \item the size of the predictor table
  \item the number of bits required to represent the spatial pattern.
\end{itemize}

Large region sizes effectively reduce the number of regions in the working set and require a smaller predictor table. However, a larger region is likely to have more blocks that exhibit varied spatial behavior and may pollute the pattern entry.  We find that going from 1KB (4096 entries) to 4KB (1024 entries) regions, the 4KB region granularity decreased miss rate by 0.3\% and increased bandwidth by 0.4\% even though both tables provide the same working set coverage (4MB).  Fixing the region size at 4KB, we studied the benefits of an unbounded table.  Compared to a 1024 entry table (\code{FINITE} in Figure~\ref{fig:Predictor_All_Apps}), the unbounded table increases miss rate by 1\% and decreases bandwidth by 0.3\% . A 1024 entry predictor table (4KB region granularity per-entry) suffices for most applications. Organizing the 1024 entries as a 128-set$\times$8-way table sufficesfor eliminating associativity related conflicts ($<$0.8\% evictions due to lack of ways).

Focusing on the number of bits required to represent the pattern table, we evaluated the use of 4-bit saturation counters (instead of 1-bit bitmaps). The saturation counters seek to avoid pattern pollution when blocks with varied spatial behavior reside in the same region. Interestingly, we find that it is more beneficial to use 1-bit bitmaps for the majority of the applications (12 out of 22); the hysteresis introduced by the counters increases training period.  

To summarize, we find that a \code{REGION} predictor with region size 4KB and 1024 entries can predict the spatial pattern in a majority of the applications. CACTI indicates that the predictor table can be indexed in 0.025ns and requires 2.3pJ per miss indexing.

\subsection{Spatial Pattern Training} 

A widely-used approach to training the predictor is to harvest the word usage information on an eviction. Unfortunately, evictions may not be frequent, which means the predictor's training period tends to be long, during which the cache performs less efficiently and/or that the application's phase has changed in the meantime. Particularly at the time of first touch (compulsory miss to a location), we need to infer the global spatial access patterns. We compare the finite region predictor (\code{FINITE} in Figure~\ref{fig:Predictor_All_Apps}) that only predicts using eviction history, against a \code{FINITE+FT}: this adds the optimization of inferring the default pattern (in this paper, from a prior run) when there is no predictor information. \code{FINITE+FT} demonstrates an avg. 1\% (max: 6\% for jbb) reduction in miss rate compared to \code{FINITE} and comes within 16\% the miss rate of \code{HISTORY}. In terms of bandwidth \code{FINITE+FT} can save 8\% of the bandwidth (up to 32\% for lbm) compared to \code{FINITE}. The percentage of first-touch accesses is shown in Table~\ref{table:Abs_Eval_Oracle}.  

% Maybe we should point out that the first touch is a more important
% optimisation for bandwidth as our fall back was to predict a full cache line
% which would have not have a first order impact on miss rate.

\begin{figure}[h]
  \centering
    \subfloat[Misses Per Kilo Instructions]{
      \includegraphics[width=\textwidth]{files/Plots/08-Predictor-MPKI.pdf}
    }
    
    \subfloat[Bandwidth]{
      \includegraphics[width=\textwidth]{files/Plots/08-Predictor-BW.pdf}
    }
    \caption[Predictor Performance]{ The charts above show the performance of the different types of \AC\ predictors in terms of misses per kilo instructions and miss bandwidth for a 64K L1 equivalent scaled to the performance of a fixed granularity conventional cache. \\ \\
      \code{FINITE}: \code{REGION} predictor (1024 entry table and 4K region size). \\
      \code{INFINITE}: Unbounded predictor table (\code{REGION} predictor). \\ 
      \code{FINITE+FT}: \code{FINITE} with hints for prediction on compulsory misses (first touches). \\
      \code{INF+FT}: \code{INFINITE} with hints for prediction on compulsory misses (first touches). \\
      \code{HISTORY}: Uses spatial pattern hints collected at eviction from a prior run. 
    }
    \label{fig:Predictor_All_Apps}
\end{figure}

\clearpage

\subsection{Predictor Summary}
\begin{itemize}
  \item For the majority of the applications (17/22) the address-region predictor with region size 4KB works well. However, five applications (apache, lbm, mcf, art, omnetpp) show better performance with PC-based indexing. For best efficiency, the predictor should adapt indexing to each application. 
  \item Updating the predictor only on evictions leads to long training periods, which causes loss in caching efficiency. We need to develop mechanisms to infer the default pattern based on global behavior demonstrated by resident cache lines.
  \item The online predictor reduces MPKI by 7\% and bandwidth by 26\% on average relative to the conventional approach. However, it still has a 14\% higher MPKI and 38\% higher bandwidth relative to the \code{HISTORY}-based predictor, indicating room for improvement in prediction. 
  \item The 1024-entry (4K region size) predictor table imposes $\simeq$0.12\% energy overhead on the overall cache hierarchy energy since it is referenced only on misses.
\end{itemize}

\section{\AC\ Adaptivity}
\label{sec:adaptivity}

We demonstrate that \AC{} can adapt better than a conventional cache to the variations in spatial locality.

\subsection{Tuning RMAX for High Spatial Locality} 

A challenge often faced by conventional caches is the desire to widen the cache block (to achieve spatial prefetching) without wasting space and bandwidth in low spatial locality applications. We study 3 specific applications: milc and tradesoap have good spatial locality and soplex has poor spatial locality. With a conventional 1M cache, when we widen the block size from 64 to 128 bytes, milc and tradesoap experience a 37\% and 39\% reduction in miss rate. However, soplex's miss rate increases by 2$\times$ and bandwidth by 3.1$\times$.

With \AC\, we do not have to make this uneasy choice as it permits \AB{}s with granularity 1---RMAX words (RMAX: maximum block size). When we increase RMAX from 64 bytes to 128 bytes, miss rate reduces by 37\% for milc and 24\% for tradesoap, while simultaneously lowering bandwidth by 7\%. Unlike the conventional cache, \AC\ is able to adapt to poor spatial locality: soplex experiences only a 2\% increase in bandwidth and 40\% increase in miss rate.

\begin{figure}[!ht]
  \centering
  \vspace{10pt}
  \includegraphics[width=0.7\textwidth]{files/Plots/08-Bar-Fixed-RMAX.pdf}
  \label{fig:rmax}
  \caption{Effect of increase in block size from 64 to 128 bytes in a 1 MB cache}
 
\end{figure}

\subsection{Predicting Strided Accesses} 
Many applications (e.g.,firefox and canneal) exhibit strided access patterns, which touch a few words in a block before accessing another block. Strided accesses patterns introduce intra-block holes (untouched words). For instance, canneal accesses $\simeq$10K distinct fixed granularity cache blocks  with a specific access pattern, \textbf{[{-}{-}x{-}{-}x{-}{-}]} (\textbf{x} indicates $i^{th}$ word has been touched). 

%!TEX root=/home/ska124/Dropbox/Thesis/thes-full.tex
\begin{table}[!htb]
  \centering
  \begin{tabular}{|l|c|c|c|c| }
    \hline
    & \multicolumn{2}{c|}{canneal} &  \multicolumn{2}{c|}{firefox} \\
    \hline
    & Miss Rate & BW & Miss Rate & BW\\
    \hline
    Policy-Miss & 10.31\% & 81.2\% & 11.18\% & 47.1\%\\
    \hline
    Policy-BW & --20.08\% & 88.09\% & --13.44\% & 56.82\%\\
    \hline
    Spatial Patterns & \multicolumn{2}{c|}{\textbf{[{-}{-}x{-}{-}x{-}{-}]
        [x{-}{-}x{-}{-}{-}{-}]}} &
    \multicolumn{2}{c|}{\textbf{[{-}x{-}{-}x{-}{-}{-}]
        [x{-}{-}{-}x{-}{-}{-}]}} \\
    \hline
    \multicolumn{4}{c}{--: indicates Miss or BW higher than Fixed.}
  \end{tabular}
  \caption{Predictor Policy Comparison}
  \label{table:predictor_policy}
\end{table}

Any predictor that uses the access pattern history has two choices when an access misses on word 3 or 6 a) A miss oriented policy (Policy-Miss) may refill the entire range of words 3--6 and eliminate the secondary miss but bring in untouched words 4--5, increasing bandwidth, and b) a bandwidth focused choice (Policy- BW) that refills only the requested word but will incur more misses. Table~\ref{table:predictor_policy} compares the two contrasting policies for \AC{} (relative to a Fixed granularity baseline cache). Policy-BW saves 9\% bandwidth compared to Policy-Miss but suffer 25-30\% higher miss rate.

%!TEX root=/home/ska124/Dropbox/Thesis/thes-full.tex
\begin{table}[h]
\centering
	\begin{tabular}{|c|c|c|c|c|c|c|c|c|}
	\hline
	 & \multicolumn{2}{c|}{MPKI} &  \multicolumn{2}{c|}{BW Bytes/1K} & CPI & \multicolumn{2}{c|}{Predictor Stats}  \\
	\hline
	          & L1        & L2    & L1$\longleftrightarrow$L2 & L2$\longleftrightarrow$Mem & & FT &  \\
	          & MPKI & MPKI & Bytes/1K & Bytes/1K & Cycles/Ins. & Miss \% &  Ins/Evict\\
	\hline
	apache    &    64.9 & 19.6    &    5,000  &    2,067         &  8.3  & 0.4 &   17 \\
	art       &    133.7  &    53.0    &    5,475  &    1,425    &  16.0 & 0.0 &   9 \\
	astar     &    0.9    &    0.3     &    70     &    35       &  1.9  & 18.0 &  1,600 \\
	cactus    &    6.9    &    4.4     &    604    &    456      &  3.5  & 7.5 &   162 \\
	canne.    &    8.3    &    5.0     &    486    &    357      &  3.2  & 5.8 &   128 \\
	eclip.    &    3.6    &    $<$0.1  &    433    &    $<$1     &  1.8  & 0.1 &   198 \\
	faces.    &    5.5    &    4.7     &    683    &    632      &  3.0  & 41.2 &  190 \\
	ferre.    &    6.8    &    1.4     &    827    &    83       &  2.1  & 1.3 &   156 \\
	firef.    &    1.5    &    1.0     &    123    &    95       &  2.1  & 11.1 &  727 \\
	fluid.    &    1.7    &    1.4     &    138    &    127      &  1.9  & 39.2 &  629 \\
	freqm.    &    1.1    &    0.6     &    89     &    65       &  2.3  & 17.7 &  994 \\
	h2        &    4.6    &    0.4     &    328    &    46       &  1.8  & 1.7 &   154 \\
	jbb       &    24.6   &    9.6     &    1,542  &    830      &  5.0  & 10.2 &  42 \\
	lbm       &    63.1   &    42.2    &    3,755  &    3,438    &  13.6 & 6.7 &   18 \\
	mcf       &    55.8   &    40.7    &    2,519  &    2,073    &  13.2 & 0.0 &   19 \\
	milc      &    16.1   &    16.0    &    1,486  &    1,476    &  6.0  & 2.4 &   66 \\
	omnet.    &    2.5    &    $<$0.1  &    158    &    $<$1     &  1.9  & 0.0 &   458 \\
	sople.    &    30.7   &    4.0     &    1,045  &    292      &  3.1  & 0.9 &   35 \\
	tpcc      &    5.4    &    0.5     &    438    &    36       &  2.0  & 0.4 &   200 \\
	trade.    &    3.6    &    $<$0.1  &    410    &    6        &  1.8  & 0.6 &   194 \\
	twolf     &    23.3   &    0.6     &    1,326  &    45       &  2.2  & 0.0 &   49 \\
	x264      &    4.1    &    1.8     &    270    &    190      &  2.2  & 12.4 &  274 \\
	\hline                                  
	\end{tabular}                                                                       
                                                                       
\caption[Absolute performance statistics]{Abslute performance statistics for the \AC\ using a \code{Region} predictor(infinite table size) with predictions for compulsory misses serviced using data collected from a prior run of the application. Higher value for Ins/Evict indicates predictor training takes longer. \\ \\
	\textbf{MPKI} : Misses / 1K instructions. \\
	\textbf{BW}: Number of words / 1K instructions. \\
	\textbf{CPI}: Clock cycles per instruction. \\
	\textbf{FT}: Percentage of misses that are compulsory misses.  \\
	\textbf{Ins/Evict}: Number of instructions between evictions. 
}
\label{table:Abs_Eval_Oracle}                                        
\end{table}                                                          
\clearpage

% First Touch info for Inf Table
% 0.4
% 0.0
% 18.0
% 7.5
% 5.8
% 0.1
% 41.2
% 1.3
% 11.1
% 39.2
% 17.7
% 1.7
% 10.2
% 6.7
% 0.0
% 2.4
% 0.0
% 0.9
% 0.4
% 0.6
% 0.0
% 12.4

% First Touch info for 1K Table
% 16.0
% 0.0
% 20.1
% 9.5
% 44.0
% 0.1
% 76.3
% 20.1
% 44.9
% 86.6
% 19.3
% 2.7
% 55.4
% 57.0
% 13.6
% 93.2
% 0.0
% 1.9
% 5.1
% 0.6
% 0.0
% 19.0


% Instructions per Eviction data for Oracle 64K

% 17
% 9
% 1,600
% 162
% 128
% 198
% 190
% 156
% 727
% 629
% 994
% 154
% 42
% 18
% 19
% 66
% 458
% 35
% 200
% 194
% 49
% 274
%!TEX root=/home/ska124/Dropbox/Thesis/thes-full.tex

%%%%%%%%%%%%%%%%%%%%%%%%%%%%%%%%%%%%%%%%%%%%%%%%%
%
%     Chapter 6
%
%%%%%%%%%%%%%%%%%%%%%%%%%%%%%%%%%%%%%%%%%%%%%%%%

\chapter{Conclusion}
\label{chap:conclusions}

In this chapter the primary findings are summarized and describe future work motivated by these findings. 

\section{Conclusions}

This dissertation presents a novel architecture for an adaptive granularity cache memory system, the \AC{}. The \AC\ adapts to the requirements of the application to cache variable sized \AB{}s. This is achived by eliminating the tag array and storing tags and variable granularity data within the same SRAM data array. The \AC\ keeps out the unused words, thus increasing caching efficiency. Experimental evaluations show a reduced miss rate, improvement in performance and reduced consumption in energy. 

Compared to a fixed granularity cache, improves cache utilization to 90\% - 99\% for most applications, saves miss rate by up to 73\% at the L1 level and up to 88\% at the LLC level, and reduces miss bandwidth by up to 84\% at the L1 and 92\% at the LLC. Correspondingly reduces on-chip memory hierarchy energy by as much as 36\% and improves performance by as much as 50\%.

We conclude that the \AC\ is a plausible design which can be implemented with considerations discussed in \S~\ref{sec:hardware_complexity}. We believe the \AC\ is a promising step towards more flexible hardware to suit the ever changing needs of software and becomes more important with added significance of energy efficient hardware.

\section{Future Work}
The possible avenues of exploration based on the current \AC\ infrastructure are described as follows.

\subsection{Adaptive Granularity Cache Coherence}
The next logical step would be to explore cache coherence. The \AC\ simulation infrastructure is built on the GEMS-Ruby framework which has been built from the ground up to design and simulate cache coherence protocols, the 

\subsection{Online prediction}
\paragraph{Compulsory Misses}
\paragraph{Error Detection and recovery}
\subsection{Replacement Policies}
\subsection{Cache Compression}
\section{Final Thoughts}

%%%  appendices, if any
% \begin{appendices}
% %!TEX root=/home/ska124/Dropbox/Thesis/thes-full.tex
%% Copyright 1998 Pepe Kubon
%%
%% `appone.tex' --- 1st appendix for thes-full.tex, thes-short-tex from
%%                  the `csthesis' bundle
%%
%% You are allowed to distribute this file together with all files
%% mentioned in READ.ME.
%%
%% You are not allowed to modify its contents.
%%

%%%%%%%%%%%%%%%%%%%%%%%%%%%%%%%%%%%%%%%%%%%%%%%%%
%
%        Appendix 1
%
%%%%%%%%%%%%%%%%%%%%%%%%%%%%%%%%%%%%%%%%%%%%%%%%

% \chapter{}
\label{app:spacing}


% \input{files/apptwo}
% \end{appendices}

%%%%%%  bibliography
\input{files/bibl}

%%%%%%  index
% \input{files/ind}

\end{document}

